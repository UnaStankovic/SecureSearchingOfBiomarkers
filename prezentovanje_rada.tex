
% !TEX encoding = UTF-8 Unicode

\documentclass[a4paper]{article}

\usepackage{color}
\usepackage{xcolor}
\usepackage{url}
\usepackage[T2A]{fontenc} % enable Cyrillic fonts
\usepackage[utf8]{inputenc} % make weird characters work
\usepackage{amsthm}
\usepackage{amsmath}
\usepackage{graphicx}
\usepackage{graphicx}
\usepackage{amssymb}
\usepackage[english,serbian]{babel}

\usepackage[unicode]{hyperref}
\hypersetup{colorlinks,citecolor=green,filecolor=green,linkcolor=blue,urlcolor=blue}

\newtheorem{definic}{Дефиниција}
\newcommand{\norm}[1]{\left\lVert#1\right\rVert}

\begin{document}

\title{Beleške za prezentovanje seminarskog rada\\ \small{Kriptografija\\ Мatematički fakultet}}

\author{Una Stanković}
\date{09.06.2017.}
\maketitle

\section{Uvodna reč} 
Kao posledica sve bržeg razvoja tehnologija za sekvenciranje genoma javlja se poterba da se očuva bezbednost podataka posebno jer se oni sve češće čuvaju u oblaku odakle se uglavnom koriste za istraživanje. Čuvanje podataka u oblaku i brz razvoj tehnologija doprineli su tome da je moguće lako pristupiti velikim skupovima genoma što može doprineti bitnim pomacima u razvoju biomedicinskih istraživanja.\\\\
Informacije koje dobijamo iz skupova genoma se najčešće koriste u medicini, biomedicinskim istraživanjima, kao i uslugama koje se direktno pružaju korisnicima, zbog čega je veoma važno da se ovim podacima rukuje sa oprezom i da se oni obezbede od neovlašćenog pristupa i korišćenja.\\

\textit{Kako to otprilike izgleda?}
Podaci o genomima se prikupljaju i postavljaju na oblak iz različitih izvora kao što su:
\begin{itemize}
\item različiti vidovi biometrijskih autentikacija
\item medicinska analiza
\item laboratorijski rezultati 
\item razne studije i drugi.
\end{itemize}


\subsection{Homomorfna enkripcija}
Homomorfna enkripcija je pojam koji nam je veoma značajan, jer mi pomoću nje obezbeđujemo privatnost podataka, ona omogućava da izračunavanja budu prenesena u obliku šifrovanog teksta.\\\\
\textbf{Homomorfna enkripcija je vid enkripcije koji nam dozvoljava da izvodimo operacije nad enkriptovanim tekstom i dobijemo enkriptovani rezultat, koji kada dekriptujemo, isti je kao rezultat koji bismo dobili da smo operaciju primenili nad neenkriptovanim tekstom.}\\
HE je odlična za očuvanje privatnosti sekvence koju analiziramo, ali se pokazuje kao veoma nepraktična za analizu informacija kod kompletnog ljudskog genoma, zato želimo da je primenjujemo samo nad nekim delovima.

\section{Postavka problema}
\textbf{Zadatak} Mi želimo da na bezbedan način izračunamo verovatnoću genetskih bolesti kroz uparivanje skupa biomarkera sa enkriptovanim genomima koji se čuvaju u javnom oblaku.\\\\
\subsection{Biomarkeri}
Biomarker je supstanca u telu koja može da se izmeri i pruži informacije o stanju u telu pacijenta. Zamislite 2 osobe, svaka ima različiti skup molekula koji kruže u telu. Pomoću tih molekula formiramo nešto što se zove otisak, oni mogu da pokazuju razlike između dvoje ili više ljudi. Na primer, ako bismo uzeli zdravu osobu i osobu koja ima dijabetes možemo da vidimo da li je neka supstanca više zastupljena u telu bolesne osobe nego u telu zdrave i da tu informaciju kasnije koristimo kao biomarker za tu bolest. Biomarkeri nam pomažu pri dijagnostifikovanju bolesti, kao i prilikom uočavanja da li neka osoba dobro reaguje na terapiju i slično.\\\\
\subsection{Proces}
Osnovni zahtev koji imamo je da ceo proces uparivanja bude izvršen korišćenjem HE što nas dovodi do drugog uslova, a to je da serveru ne želimo da otkrivamo nikakve informacije o bazi niti o upitu.\\
Pretpostavimo da klijent ima VCF fajl (engl. Variation Call Format), koji sadrži informacije o genotipu, kao što su broj hromozoma i pozicija u genomu. Klijent enkriptuje informacije koristeći homomorfnu enkripciju i server računa tačno poklapanje nad enkriptovanim podacima. Ishod je prisustvo/odsustvo specifičnih biomarkera. Na kraju, klijent dekriptuje rezultat uz pomoć tajnog ključa homomorfne enkripcije.

\section{Pretraga i izvlačenje podataka iz baze}
Posmatrajmo bazu koja je skup od n torki (engl. tuples), gde se svaka torka sastoji iz para ($d_i$,$\alpha_i$) za i = 1,...,n, gde je $d_i$ oznaka podatka iz domena \{0,1,...,$\tau-1$\}, a $\alpha_i$ odgovarajuća vrednost atributa u prostoru običnog teksta $Z_t \backslash \{0\}$. Želimo da se sve oznake teksta razlikuju tako da u slučaju baze podataka koja sadrži informacije o nekoj osobi, $\alpha_i$ može da bude broj godina korisnika čiji je identifikacioni broj $d_i$.\\\\
Kada imamo datu oznaku upita d iz domena oznaka i vrednost upita $\alpha$ iz prostora običnog teksta, problem spajanja se svodi na određivanje postojanja indeksa \textit{i}, takvog da ($d$,$\alpha$) = ($d_i$,$\alpha_i$). Posmatrajmo sada pojednostavljen upit za pretragu: odaberi $\alpha_i$ ako postoji indeks \textit{i} takav da je $d_i$ = $d$, inače nula.\\\\

\subsection{Metod za bezbednu pretragu i izvlačenje podataka iz baze}
Osnovna ideja je korišćenje narednog metoda za enkodiranje baze koji je pogodan za efikasno izračunavanje jednakosti i izvlačenje:
	$$DB(X) = \sum_{i} \alpha_i X^{d_i} \in \mathbb{Z}$$ 
Korisnik enkriptuje polinom sa javnim ključem i šifrovani tekst čuva na serveru. U fazi ispitivanja upita, sa nazivom upita \textit{d}, korisnik enkriptuje$X^{-d}$ sa simetričnom enkripcijom, a šifrovani tekst šalje ka serveru.

\subsection{Enkodiranje informacija o genomu}
VCF fajl sadrži linije informacija o genotipu, gde se svaka od njih sastoji iz tripleta ($ch_i$,$pos_i$, SNP$s_i$) tj. broja hromozoma, pozicije i sekvence SNP alela  (koji moraju biti A, T, G ili C).\\\\
Broj hromozoma je neki ceo broj od 1 do 22,X i Y. Nenegativni ceo broj pos predstavlja referencu na poziciju sa prvom bazom koja ima poziciju jedan, a SNPs je referenca na niz \{A,T,G,C\}.\\
Upit korisnika je, takođe, triplet ovakvog oblika, pa nam je cilj da odredimo postoji li ili ne prisustvo odgovarajućeg biomarkera u fajlu iz baze. \\\\
Postavićemo $n_SNP$ za maksimalni broj alela za vršenje poređenja između genoma upita i korisničkog genoma. Svaki od SNP-ova je predstavljen sa dva bita tako da 
$$A \rightarrow 00, T \rightarrow 01, G \rightarrow 10, C \rightarrow 11$$
a potom su ti bitovi konkatenirani. Potom stavljamo bit 1 na početak stringa sa leve strane da označimo početnu poziciju. I na kraju popunjavamo string dužine $\textit{l}_{SNP}$ = 2$n_{SNP}$ + 1 i konvertujemo dobijeno u ceo broj, označen sa $\alpha_i$. Ako se desi da je neki nukleotid nepoznat u datom delu, onda takav string enkodiramo nulama. Na primer: $'$GC$'$ je enkodiran binarno kao 11011, što je ceo broj 27.

\section{Zaključak}
	Sa razvojem bioinformatike i sve većim potrebama za čuvanjem podataka u oblaku, možemo očekivati porast u broju, kvalitetu i pouzdanosti algoritama za enkripciju podataka, kao i razvoj sve boljih i bržih mehanizama za pretragu genoma.

\end{document}